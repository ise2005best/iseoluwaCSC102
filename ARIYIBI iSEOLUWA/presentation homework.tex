\documentclass{article}
\usepackage{graphicx}

\begin{document}
 
 	\textbf{Historical Perspective of 5 different Programming Language}
 \begin{center}
 		\textbf{30th October 2021}
 	
 	\textbf{Ariyibi Iseoluwa}
 \end{center}
\newpage
	\centering
		\begin{figure}[h]
			\includegraphics[width=0.5\linewidth]{java}
			\centering
		\end{figure}
		JAVA	
	\section{Brief History Of Java}

	\ Java was developed by James Gosling in 1995. He is known as the father of Java. James Gosling, Mike Sheridan and Patrick Naughton started the Java language project in 1991. They were called the ‘Green Team’.
	\subsection{Application examples that exist in Java}
	\begin{itemize}
		\item Mobile Applications
		\item Business Applications
		\item Gaming Applications
		\item Big Data Technologies
		\item Desktop GUI Applications
	\end{itemize}
\subsubsection{Available Integrated Development environment (IDE’s) in Java}
\begin{itemize}
	\item Eclipse
	\item Blue J
	\item Xcode
	\item NetBeans
	\item GreenFoot
	\item Myeclipse
\end{itemize}
\paragraph*{Related Programming Languages}
Ruby is very similar to java. They are both object oriented programing languages. They are also both general purpose programming language. 
\newpage
\begin{center}
	HTML
	\begin{figure}[h]
		\includegraphics[width=0.4\linewidth]{../../Downloads/html}
		\label{fig:html}
	\end{figure}
\end{center}
\section{Brief History of HTML}
\ HTML was created by Tim Berners-Lee, a physicist from Switzerland. He invented HTML in 1991. HTML was later published in 1995. It was called HTML 2.0. It was the standard language for designing and creating websites.
\subsection{Application examples that exist in HTML}
\begin{itemize}
	\item Web page development
	\item Web document creation
	\item Internet navigation
\end{itemize}
\subsubsection{Available Integrated Development environment (IDE’s) in HTML}
\begin{itemize}
	\item NetBeans
	\item Brackets
	\item PyCharm
	\item RubyMine
	\item Atom by Github
\end{itemize}
\paragraph{Related Programming Language}
\ CSS is a complementary language to HTML. It is used to style HTML markup code. A page’s CSS styles are called HTML tags.
\newpage
\begin{center}
	C++
\begin{figure}[h]
	\includegraphics[width=0.4\linewidth]{../../Downloads/bjarne}
	\caption{Bjarne Stroustrup}
	\label{Founder of C++}
\end{figure}
\end{center}
\section{Brief Histroy of C++}
\ C++ was developed by Bjarne Stroustrup from Bell laboratories in 1979. It was first called ‘c with classes’ before it was changed to ‘C++’ in 1983. It was first made available in 1985. 
\ C++ was designed to be an efficient and flexible language similar to C.
\subsection{Application examples that exist in C++}
\begin{itemize}
	\item Gaming Applications
	\item Browsers
	\item Banking Application
	\item Database Software
	\item GUI based applications
\end{itemize}
\paragraph{Related Programming Language}
\begin{itemize}
	\item Ruby
	\item Java programming
	\item Python programming
\end{itemize}
\newpage
\begin{center}
	JAVASCRIPT
\begin{figure}[h]
	\includegraphics[width=0.4\linewidth]{../../Downloads/javascript}
	\label{fig:javascript}
\end{figure}
\end{center}
\section{Brief History Of Javascript}
\ Javascript was invented by Brendan Eich in 1995. Its original name is called Mocha. It is a Highlevel Language.
\subsection{Application examples that exist in Javascript}
\begin{itemize}
	\item Website
	\item Art
	\item Games
	\item Web applications
	\item Web servers
	\item Smartwatch Applications
\end{itemize} 
\subsubsection{Available Integrated Development environment (IDE’s)in JavaScript}
\begin{itemize}
	\item Eclipse
	\item NetBeans
	\item Atorm
	\item Sourceliar
	\item AWS Cloud 9
\end{itemize}
\paragraph{Related Programming Language}
\ Scala.js translates the scala programming language to Java script. Javascript code and libraries are compatible with scala code.
\newpage
\begin{center}
	FORTRAN
\begin{figure}[h]
	\includegraphics[width=0.4\linewidth]{../../Downloads/fortran}
	\label{fig:fortran}
\end{figure}
\end{center}
\section{Brief History Of Fortran}
\ The full name of FORTRAN is ( Formula Translation). It was created by John Backus in 1957. It is one of the oldest programming language. It was designed to allow easy translation of math formula into code.
It was the first high level language. Fortran is one of the easiest programming language to learn.
\subsection{Application Examples of Fortran}
\begin{itemize}
  \item Analysis of scientific data
\end{itemize}
\subsubsection{Available Integrated Development environment (IDE’s)in Fortran}
\begin{itemize}
	\item NetBeans
	\item Oracle Developer Studio
	\item Eclipse Photran
\end{itemize}
\newpage
\paragraph{THANKS FOR LISTENING}
\end{document}

